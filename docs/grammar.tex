The LARS code generator presented here accepts as input a sequence of stencil
statements that are expressed in a restricted subset of C consistent with the
grammar described in figure \ref{fig:start-grammar}. The input to the code
generator is a full stencil code in a .c file, with the regions that the user
wants to reorder annotated with \texttt{\#pragma begin stencil$\ldots$} and a
matching \texttt{\#pragma end}. The parser is simplified and does not perform
any syntactic verification, so when creating a new benchmark, it is the user's
responsibility to ensure that the syntax adheres to the specifications for the
code generator to work correctly. 

We begin by defining $\mathit{Parameter}$ and $\mathit{Iterator}$ in context of
the computation. $\mathit{Parameter}$ represents the program parameters, and
$\mathit{Iterator}$ represents the loop iterators. Even though the values of
the parameters are unknown at compile time, they are fixed-value, i.e., their
values are not changed by the statements in the stencil function. We also
assume the iterator is only modified by the increment expression of the loop,
and not by the stencil statements.
 
In the grammar of figure \ref{fig:start-grammar}, a $\mathit{BaseExpr}$ in a
statement represents an accessed storage location. A $\mathit{BaseExpr}$ can be
an array access, or a scalar. The first restriction we place on the index
expressions of the array accesses ($\mathit{AccessExpr}$) is that they must be
a function of the parameters, iterators, and integers (all immutable within the
stencil function body).

A $\mathit{BaseExpr}$ is identified by a \textit{label}. A crucial part of the
reordering framework is to identify the \textit{same} $\mathit{BaseExpr}$. This
is done by checking the equivalence of the labels that are assigned to any two
$\mathit{BaseExpr}$. This is complicated by the fact that with unrolling, one
has to guarantee that \texttt{A[i+1+1]} will be assigned the same label as
\texttt{A[i+2]}. In general, the array index expressions can be rewritten in
multiple, albeit equivalent ways -- \texttt{A[i+j]} is the same as
\texttt{A[j+i]}, \texttt{A[i+1+j]} is the same as \texttt{A[i+j+1]}, and so on.
To ensure that the semantically equivalent $\mathit{BaseExpr}$ are assigned the
same label, the code generator has a mechanism that ``canonicalizes'' the
accesses, so that semantically equivalent accesses have the same pattern. 

\begin{figure}[t!]    
\footnotesize
  \begin{align}
    \hline
    \langle\text{Program}\rangle ::=& \;  \langle\text{StmtList}\rangle {\color{tablecolor!60}\texttt{;}} \nonumber \\
    \langle\text{StmtList}\rangle ::=& \; \langle\text{Stmt}\rangle \langle\text{StmtList}\rangle \; | \; \langle\text{Stmt}\rangle \nonumber \\
    \langle\text{Stmt}\rangle ::=& \; \langle\text{BaseExpr}\rangle \langle\text{AssignOp}\rangle \langle\text{Expr}\rangle {\color{tablecolor!60}\texttt{;}} \nonumber \\
    \langle\text{Expr}\rangle ::=& \; \text{Side-effect-free expression of~} \langle\text{BaseExpr}\rangle \nonumber \\
    \langle\text{BaseExpr}\rangle ::=& \; \langle\text{ID}\rangle \; | \; \langle\text{ArrayExpr}\rangle \nonumber \\
    \langle\text{ArrayExpr}\rangle ::=& \; \langle\text{ID}\rangle \langle\text{ArrayDims}\rangle \nonumber \\
    \langle\text{ArrayDims}\rangle ::=& \; {\color{tablecolor!60}\texttt{[}}\langle\text{AccessExpr}\rangle {\color{tablecolor!60}\texttt{]}} \langle\text{ArrayDims}\rangle \; | \; {\color{tablecolor!60}\texttt{[}}\langle\text{AccessExpr}\rangle {\color{tablecolor!60}\texttt{]}} \nonumber \\
    \langle\text{AccessExpr}\rangle ::=& \; \text{Side-effect-free expression of~} \langle\text{AccessElement}\rangle \text{ with restrictions described in text}\nonumber \\
	\langle\text{AccessElement}\rangle ::=& \; \langle\text{Iterator}\rangle \; | \; \langle\text{Parameter}\rangle \; | \; \langle\text{Const}\rangle \nonumber \\
    \langle\text{AssignOp}\rangle ::=& \; {\color{tablecolor!60}\texttt{=}} \; | \; {\color{tablecolor!60}\texttt{+=}} \; | \; {\color{tablecolor!60}\texttt{-=}} \; | \; {\color{tablecolor!60}\texttt{*=}} \; | \; \ldots \nonumber  \\
    \langle\text{DataType}\rangle ::=& \; {\color{tablecolor!60}\texttt{double}} \; | \; {\color{tablecolor!60}\texttt{float}} \; | \; {\color{tablecolor!60}\texttt{int}} \; | \; {\color{tablecolor!60}\texttt{bool}} \; | \; \ldots \nonumber  \\
    \hline \nonumber
  \end{align}
\vspace*{-0.6cm}
  \caption{Grammar for valid input statements to the code generator}
\label{fig:start-grammar}
\end{figure}


\begin{figure}[t!]
  \footnotesize
  \begin{align}
    \hline
    \langle\text{CodeBody}\rangle ::=& \;  \langle\text{StmtList$_l$}\rangle {\color{tablecolor!60}\texttt{;}} \nonumber \\
    \langle\text{StmtList$_l$}\rangle ::=& \; \langle\text{Assignment$_l$}\rangle {\color{tablecolor!60}\texttt{;}} \langle\text{StmtList$_l$}\rangle \; | \; \langle\text{Assignment$_l$}\rangle \nonumber \\
    \langle\text{Assignment$_l$}\rangle ::=& \; \langle\text{BaseExpr}\rangle \langle\text{AssignOp}\rangle \langle\text{Expr$_l$}\rangle \nonumber \\
    \langle\text{Expr$_l$}\rangle ::=& \; \langle\text{BaseExpr}\rangle \langle\text{BinaryOp$_l$}\rangle \langle\text{BaseExpr}\rangle \; | \; \langle\text{BaseExpr}\rangle \nonumber \\
    \langle\text{BinaryOp$_l$}\rangle ::=& \; {\color{tablecolor!60}\texttt{+}} \; | \; {\color{tablecolor!60}\texttt{-}} \; | \; {\color{tablecolor!60}\texttt{*}} \; | \; {\color{tablecolor!60}\texttt{/}} \; | \; {\color{tablecolor!60}\texttt{|}} \; | \; {\color{tablecolor!60}\texttt{\&}} \; | \; \ldots \nonumber  \\
    \hline \nonumber
  \end{align}
   \vspace*{-0.6cm}
  \caption{Grammar for the lowered 3-address IR for instruction reordering}
\label{fig:lower-grammar}
\end{figure}


We can rewrite an array index expression so that it comprises an
string-access component (e.g., \texttt{(i+j)} in access \texttt{A[i+j+2]}),
and an integer offset component (e.g., 2 in accesses \texttt{A[i+j+1+1]} and
\texttt{A[i+j+2]}).  Within the code generator, the logic that separates the
index expression into the two components is encoded in the function
\textit{array\_access\_info ()}, and it works in two steps: 


\begin{compactenum}
\item it separates the string-access component and the integer offset from an
array index expression
\item it then canonicalizes both the extracted components. For the
string-access component, this involves identifying the presence of an operator
in the string, and then exploiting the properties of the detected operator to
align the individual iterators/parameters in lexicographical order. For
example, \texttt{(j-i)} will be converted to \texttt{(j+(-i))}, and then
\texttt{(j \textit{op} i)} will be rearranged as \texttt{(i \textit{op} j)}, iff
\texttt{i} precedes \texttt{j} in lexicographical ordering, and \textit{op} is
commutative, like +,*. For the offset component, it identifies the operators
involved, and performs the mathematical operation to reduce the offset
component to a single integer value.
\end{compactenum}

This canonicalization is non-trivial, and currently primitive in implementation
in the code generator. This necessitates a second restriction on the index
expressions of the arrays: at present, we restrict the number of operators in
the string-access component of an index expression to 1. For example, we allow
A[k+1+1][2*j+2][i] since we can canonicalize each of the array index
expressions, but not something as complex as A[(k+2)*(k+3)][j][i], since this
access will result in an string-offset component (k*k+5k) where there are two
arithmetic operations involved.
 
In summary, we impose the following restrictions on the construct of statements:
\begin{itemize}
\item The statements must not modify the iterators (e.g., $i$, $j$ and $k$ in listing
\ref{listing:new-eg})
\item The statements can only have mathematical functions like $\mathit{sine}$,
$\mathit{sqrt}$, etc. on the RHS, that are side-effect free.  
\item The accesses in the stencil statements are either scalars or array
elements. The array index expressions are functions of the iterators,
parameters, and integers, with simple arithmetic operations between them
(+,-,*,/). 
\item Furthermore, we allow only \textit{one} arithmetic operation in the
string-access component of the array index expressions, and the offset
component must be an integer.
\end{itemize} 

Once the parsing is complete, the code
generator lowers the input further to sub-statements consistent with the
grammar of figure \ref{fig:lower-grammar}. Informally, this lowering is like
GIMPLE IR of GCC \cite{gcc} with accumulations. 


\section{Creating an example}
\begin{figure}[t!]
\vspace*{-1em}
%\vspace*{-15pt}
\begin{lstlisting}[escapechar=@,caption={The input representation with pragmas},label={listing:new-eg}]
  for (k=1; k<=N-2; k++) {
       for (j=1; j<=N-2; j++) {
           for (i=1; i<=N-2; i++) {
#pragma begin stencil1 unroll k=4,j=1,i=1
                   out[k][j][i] = $a$*(in[k+1][j][i]) + $b$*(in[k][j-1][i] +
                              in[k][j][i-1] + in[k][j][i] + in[k][j][i+1] +
                                  in[k][j+1][i]) + $c$*(in[k-1][j][i]);
#pragma end stencil1
           }
       }
  }
\end{lstlisting}
\end{figure}

Listing \ref{listing:new-eg} shows a snippet of a C input code that can be
passed as an input to the reordering framework. The computation in lines 5-7 is
affine, and in compliance with the restrictions we placed on the input grammar.
Note the pragmas in line 4 and 8 that surround the statement -- these demarcate
the region that the code generator can parse and optimize. Additionally, the
pragma supplies the unrolling factor for each iterator. Given an input file
with several such pragma-demarcated regions, the steps in optimizing the code
will be: 

\begin{enumerate} 
\item A script will remove the statements between all the pragma, and put them
into separate \texttt{.idsl} files, each named based on the stencil name (e.g.,
\texttt{stencil1.idsl} for the example in listing \ref{listing:new-eg}).
Therefore, each stencil name must be distinct.
\item From the pool of \texttt{stencilname.idsl} files generated, the code
generator will pick one at a time, parse it, optimize it, and write the results
back in \texttt{stencilname.c}.  
\item A script will then read the contents from \texttt{stencilname.c}, and
replace the original statements demarcated by \texttt{\#pragma begin
stencilname $\ldots$ \#pragma end stencilname} with the optimized code. 
\end{enumerate} 
The changes are made in-place to the input C file.
Please look at the benchmarks in the examples folder to learn more about how to use this
code generator for your applications. 
 

